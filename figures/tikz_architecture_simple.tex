% TECM Framework - Simplified Version for Paper Inclusion
% Include this in main.tex with: % TECM Framework - Simplified Version for Paper Inclusion
% Include this in main.tex with: % TECM Framework - Simplified Version for Paper Inclusion
% Include this in main.tex with: % TECM Framework - Simplified Version for Paper Inclusion
% Include this in main.tex with: \input{figures/tikz_architecture_simple}
% Requires: \usepackage{tikz}, \usetikzlibrary{shapes, arrows.meta, positioning, fit, calc}

\begin{figure}[H]
\centering
\resizebox{\textwidth}{!}{%
\begin{tikzpicture}[
    model/.style={rectangle, rounded corners=5pt, minimum height=2.2cm, minimum width=3.8cm,
                  font=\small, align=center, line width=1.2pt},
    state/.style={rectangle, rounded corners=4pt, draw=gray!70, fill=gray!8,
                  minimum height=1.4cm, minimum width=11cm, font=\small, align=center, line width=1.5pt},
    tipping/.style={rectangle, rounded corners=5pt, draw=orange!80!black, fill=yellow!15,
                    minimum height=1.2cm, minimum width=8cm, font=\small, align=center, line width=1.5pt},
    arrow/.style={-{Stealth[length=3mm]}, line width=1.2pt},
    feedback/.style={-{Stealth[length=2.5mm]}, line width=1pt, dashed, red!60!black},
]

% Title
\node[font=\bfseries\large] at (6, 9) {TECM Framework Architecture};

% Model I
\node[model, draw=orange!70, fill=orange!10] (m1) at (0, 6.5) {
    \textbf{Model I}\\[2pt]
    Task Decomposition\\
    \& Risk Assessment\\[4pt]
    $\Xi_o = \sum_k w_k \xi_k (1-\alpha_p p_k)$\\[2pt]
    $P_i$ = Physical Protection
};

% Model II
\node[model, draw=green!60!black, fill=green!8] (m2) at (6, 6.5) {
    \textbf{Model II}\\[2pt]
    VBC Dynamics\\[4pt]
    $\frac{dA}{dt}$, $\frac{dE}{dt}$\\[2pt]
    $\rho(t)$ = Congestion\\
    $D(t)$ = Demand (Jevons)
};

% Model III
\node[model, draw=purple!60, fill=purple!8] (m3) at (12, 6.5) {
    \textbf{Model III}\\[2pt]
    Multi-Channel Learning\\[4pt]
    $\frac{dV_s}{dt} = L_{job} + L_{soc}$\\[2pt]
    $+ L_{auto} - \delta_V V_s$\\
    Educational Feedback
};

% State Variables
\node[state] (state) at (6, 3.5) {
    \textbf{Shared State Variables}\\[4pt]
    $A(t)$ Adoption \quad | \quad $E(t)$ Employment \quad | \quad $V_s(t)$ Capability \quad | \quad $\rho(t)$ Congestion \quad | \quad $D(t)$ Demand
};

% Tipping Point
\node[tipping] (tip) at (6, 1) {
    \textbf{Tipping Point Analysis}\\[4pt]
    $A^*(t) = \frac{1}{2 \cdot s(t)}$ \quad where \quad $s(t) = \frac{s_{base}}{1+\beta_s V_s(t)}$
};

% Direct arrows (Model to State)
\draw[arrow] (m1.south) -- ++(0,-0.5) -| ($(state.north west)+(1.5,0)$);
\draw[arrow] (m2.south) -- (state.north);
\draw[arrow] (m3.south) -- ++(0,-0.5) -| ($(state.north east)+(-1.5,0)$);

% State to Tipping
\draw[arrow] (state.south) -- (tip.north);

% Feedback loops
\draw[feedback] ($(state.east)+(0,0.2)$) -- ++(1,0) |- ($(m3.east)+(0,-0.3)$);
\node[font=\scriptsize, text=red!60!black] at (13.8, 5) {$V_s \to s\downarrow$};

\draw[feedback] ($(state.west)+(0,0.2)$) -- ++(-1,0) |- ($(m1.west)+(0,-0.3)$);
\node[font=\scriptsize, text=red!60!black] at (-1.8, 5) {$\Xi_o, P_i$};

\draw[feedback] (tip.east) -- ++(0.5,0) |- ($(m2.east)+(0,-0.5)$);
\node[font=\scriptsize, text=red!60!black] at (11, 2.5) {$A^* \to E$};

% Legend
\draw[arrow] (0.5, -0.3) -- (1.5, -0.3);
\node[font=\scriptsize] at (2.5, -0.3) {Direct Effect};
\draw[feedback] (4, -0.3) -- (5, -0.3);
\node[font=\scriptsize] at (6.2, -0.3) {Feedback Loop};

\end{tikzpicture}
}
\caption{TECM Framework Architecture. Three interconnected models share state variables and create feedback loops. The demand rebound (Jevons Paradox) and educational feedback loops are key mechanisms enabling intervention.}
\label{fig:architecture_new}
\end{figure}

% Requires: \usepackage{tikz}, \usetikzlibrary{shapes, arrows.meta, positioning, fit, calc}

\begin{figure}[H]
\centering
\resizebox{\textwidth}{!}{%
\begin{tikzpicture}[
    model/.style={rectangle, rounded corners=5pt, minimum height=2.2cm, minimum width=3.8cm,
                  font=\small, align=center, line width=1.2pt},
    state/.style={rectangle, rounded corners=4pt, draw=gray!70, fill=gray!8,
                  minimum height=1.4cm, minimum width=11cm, font=\small, align=center, line width=1.5pt},
    tipping/.style={rectangle, rounded corners=5pt, draw=orange!80!black, fill=yellow!15,
                    minimum height=1.2cm, minimum width=8cm, font=\small, align=center, line width=1.5pt},
    arrow/.style={-{Stealth[length=3mm]}, line width=1.2pt},
    feedback/.style={-{Stealth[length=2.5mm]}, line width=1pt, dashed, red!60!black},
]

% Title
\node[font=\bfseries\large] at (6, 9) {TECM Framework Architecture};

% Model I
\node[model, draw=orange!70, fill=orange!10] (m1) at (0, 6.5) {
    \textbf{Model I}\\[2pt]
    Task Decomposition\\
    \& Risk Assessment\\[4pt]
    $\Xi_o = \sum_k w_k \xi_k (1-\alpha_p p_k)$\\[2pt]
    $P_i$ = Physical Protection
};

% Model II
\node[model, draw=green!60!black, fill=green!8] (m2) at (6, 6.5) {
    \textbf{Model II}\\[2pt]
    VBC Dynamics\\[4pt]
    $\frac{dA}{dt}$, $\frac{dE}{dt}$\\[2pt]
    $\rho(t)$ = Congestion\\
    $D(t)$ = Demand (Jevons)
};

% Model III
\node[model, draw=purple!60, fill=purple!8] (m3) at (12, 6.5) {
    \textbf{Model III}\\[2pt]
    Multi-Channel Learning\\[4pt]
    $\frac{dV_s}{dt} = L_{job} + L_{soc}$\\[2pt]
    $+ L_{auto} - \delta_V V_s$\\
    Educational Feedback
};

% State Variables
\node[state] (state) at (6, 3.5) {
    \textbf{Shared State Variables}\\[4pt]
    $A(t)$ Adoption \quad | \quad $E(t)$ Employment \quad | \quad $V_s(t)$ Capability \quad | \quad $\rho(t)$ Congestion \quad | \quad $D(t)$ Demand
};

% Tipping Point
\node[tipping] (tip) at (6, 1) {
    \textbf{Tipping Point Analysis}\\[4pt]
    $A^*(t) = \frac{1}{2 \cdot s(t)}$ \quad where \quad $s(t) = \frac{s_{base}}{1+\beta_s V_s(t)}$
};

% Direct arrows (Model to State)
\draw[arrow] (m1.south) -- ++(0,-0.5) -| ($(state.north west)+(1.5,0)$);
\draw[arrow] (m2.south) -- (state.north);
\draw[arrow] (m3.south) -- ++(0,-0.5) -| ($(state.north east)+(-1.5,0)$);

% State to Tipping
\draw[arrow] (state.south) -- (tip.north);

% Feedback loops
\draw[feedback] ($(state.east)+(0,0.2)$) -- ++(1,0) |- ($(m3.east)+(0,-0.3)$);
\node[font=\scriptsize, text=red!60!black] at (13.8, 5) {$V_s \to s\downarrow$};

\draw[feedback] ($(state.west)+(0,0.2)$) -- ++(-1,0) |- ($(m1.west)+(0,-0.3)$);
\node[font=\scriptsize, text=red!60!black] at (-1.8, 5) {$\Xi_o, P_i$};

\draw[feedback] (tip.east) -- ++(0.5,0) |- ($(m2.east)+(0,-0.5)$);
\node[font=\scriptsize, text=red!60!black] at (11, 2.5) {$A^* \to E$};

% Legend
\draw[arrow] (0.5, -0.3) -- (1.5, -0.3);
\node[font=\scriptsize] at (2.5, -0.3) {Direct Effect};
\draw[feedback] (4, -0.3) -- (5, -0.3);
\node[font=\scriptsize] at (6.2, -0.3) {Feedback Loop};

\end{tikzpicture}
}
\caption{TECM Framework Architecture. Three interconnected models share state variables and create feedback loops. The demand rebound (Jevons Paradox) and educational feedback loops are key mechanisms enabling intervention.}
\label{fig:architecture_new}
\end{figure}

% Requires: \usepackage{tikz}, \usetikzlibrary{shapes, arrows.meta, positioning, fit, calc}

\begin{figure}[H]
\centering
\resizebox{\textwidth}{!}{%
\begin{tikzpicture}[
    model/.style={rectangle, rounded corners=5pt, minimum height=2.2cm, minimum width=3.8cm,
                  font=\small, align=center, line width=1.2pt},
    state/.style={rectangle, rounded corners=4pt, draw=gray!70, fill=gray!8,
                  minimum height=1.4cm, minimum width=11cm, font=\small, align=center, line width=1.5pt},
    tipping/.style={rectangle, rounded corners=5pt, draw=orange!80!black, fill=yellow!15,
                    minimum height=1.2cm, minimum width=8cm, font=\small, align=center, line width=1.5pt},
    arrow/.style={-{Stealth[length=3mm]}, line width=1.2pt},
    feedback/.style={-{Stealth[length=2.5mm]}, line width=1pt, dashed, red!60!black},
]

% Title
\node[font=\bfseries\large] at (6, 9) {TECM Framework Architecture};

% Model I
\node[model, draw=orange!70, fill=orange!10] (m1) at (0, 6.5) {
    \textbf{Model I}\\[2pt]
    Task Decomposition\\
    \& Risk Assessment\\[4pt]
    $\Xi_o = \sum_k w_k \xi_k (1-\alpha_p p_k)$\\[2pt]
    $P_i$ = Physical Protection
};

% Model II
\node[model, draw=green!60!black, fill=green!8] (m2) at (6, 6.5) {
    \textbf{Model II}\\[2pt]
    VBC Dynamics\\[4pt]
    $\frac{dA}{dt}$, $\frac{dE}{dt}$\\[2pt]
    $\rho(t)$ = Congestion\\
    $D(t)$ = Demand (Jevons)
};

% Model III
\node[model, draw=purple!60, fill=purple!8] (m3) at (12, 6.5) {
    \textbf{Model III}\\[2pt]
    Multi-Channel Learning\\[4pt]
    $\frac{dV_s}{dt} = L_{job} + L_{soc}$\\[2pt]
    $+ L_{auto} - \delta_V V_s$\\
    Educational Feedback
};

% State Variables
\node[state] (state) at (6, 3.5) {
    \textbf{Shared State Variables}\\[4pt]
    $A(t)$ Adoption \quad | \quad $E(t)$ Employment \quad | \quad $V_s(t)$ Capability \quad | \quad $\rho(t)$ Congestion \quad | \quad $D(t)$ Demand
};

% Tipping Point
\node[tipping] (tip) at (6, 1) {
    \textbf{Tipping Point Analysis}\\[4pt]
    $A^*(t) = \frac{1}{2 \cdot s(t)}$ \quad where \quad $s(t) = \frac{s_{base}}{1+\beta_s V_s(t)}$
};

% Direct arrows (Model to State)
\draw[arrow] (m1.south) -- ++(0,-0.5) -| ($(state.north west)+(1.5,0)$);
\draw[arrow] (m2.south) -- (state.north);
\draw[arrow] (m3.south) -- ++(0,-0.5) -| ($(state.north east)+(-1.5,0)$);

% State to Tipping
\draw[arrow] (state.south) -- (tip.north);

% Feedback loops
\draw[feedback] ($(state.east)+(0,0.2)$) -- ++(1,0) |- ($(m3.east)+(0,-0.3)$);
\node[font=\scriptsize, text=red!60!black] at (13.8, 5) {$V_s \to s\downarrow$};

\draw[feedback] ($(state.west)+(0,0.2)$) -- ++(-1,0) |- ($(m1.west)+(0,-0.3)$);
\node[font=\scriptsize, text=red!60!black] at (-1.8, 5) {$\Xi_o, P_i$};

\draw[feedback] (tip.east) -- ++(0.5,0) |- ($(m2.east)+(0,-0.5)$);
\node[font=\scriptsize, text=red!60!black] at (11, 2.5) {$A^* \to E$};

% Legend
\draw[arrow] (0.5, -0.3) -- (1.5, -0.3);
\node[font=\scriptsize] at (2.5, -0.3) {Direct Effect};
\draw[feedback] (4, -0.3) -- (5, -0.3);
\node[font=\scriptsize] at (6.2, -0.3) {Feedback Loop};

\end{tikzpicture}
}
\caption{TECM Framework Architecture. Three interconnected models share state variables and create feedback loops. The demand rebound (Jevons Paradox) and educational feedback loops are key mechanisms enabling intervention.}
\label{fig:architecture_new}
\end{figure}

% Requires: \usepackage{tikz}, \usetikzlibrary{shapes, arrows.meta, positioning, fit, calc}

\begin{figure}[H]
\centering
\resizebox{\textwidth}{!}{%
\begin{tikzpicture}[
    model/.style={rectangle, rounded corners=5pt, minimum height=2.2cm, minimum width=3.8cm,
                  font=\small, align=center, line width=1.2pt},
    state/.style={rectangle, rounded corners=4pt, draw=gray!70, fill=gray!8,
                  minimum height=1.4cm, minimum width=11cm, font=\small, align=center, line width=1.5pt},
    tipping/.style={rectangle, rounded corners=5pt, draw=orange!80!black, fill=yellow!15,
                    minimum height=1.2cm, minimum width=8cm, font=\small, align=center, line width=1.5pt},
    arrow/.style={-{Stealth[length=3mm]}, line width=1.2pt},
    feedback/.style={-{Stealth[length=2.5mm]}, line width=1pt, dashed, red!60!black},
]

% Title
\node[font=\bfseries\large] at (6, 9) {TECM Framework Architecture};

% Model I
\node[model, draw=orange!70, fill=orange!10] (m1) at (0, 6.5) {
    \textbf{Model I}\\[2pt]
    Task Decomposition\\
    \& Risk Assessment\\[4pt]
    $\Xi_o = \sum_k w_k \xi_k (1-\alpha_p p_k)$\\[2pt]
    $P_i$ = Physical Protection
};

% Model II
\node[model, draw=green!60!black, fill=green!8] (m2) at (6, 6.5) {
    \textbf{Model II}\\[2pt]
    VBC Dynamics\\[4pt]
    $\frac{dA}{dt}$, $\frac{dE}{dt}$\\[2pt]
    $\rho(t)$ = Congestion\\
    $D(t)$ = Demand (Jevons)
};

% Model III
\node[model, draw=purple!60, fill=purple!8] (m3) at (12, 6.5) {
    \textbf{Model III}\\[2pt]
    Multi-Channel Learning\\[4pt]
    $\frac{dV_s}{dt} = L_{job} + L_{soc}$\\[2pt]
    $+ L_{auto} - \delta_V V_s$\\
    Educational Feedback
};

% State Variables
\node[state] (state) at (6, 3.5) {
    \textbf{Shared State Variables}\\[4pt]
    $A(t)$ Adoption \quad | \quad $E(t)$ Employment \quad | \quad $V_s(t)$ Capability \quad | \quad $\rho(t)$ Congestion \quad | \quad $D(t)$ Demand
};

% Tipping Point
\node[tipping] (tip) at (6, 1) {
    \textbf{Tipping Point Analysis}\\[4pt]
    $A^*(t) = \frac{1}{2 \cdot s(t)}$ \quad where \quad $s(t) = \frac{s_{base}}{1+\beta_s V_s(t)}$
};

% Direct arrows (Model to State)
\draw[arrow] (m1.south) -- ++(0,-0.5) -| ($(state.north west)+(1.5,0)$);
\draw[arrow] (m2.south) -- (state.north);
\draw[arrow] (m3.south) -- ++(0,-0.5) -| ($(state.north east)+(-1.5,0)$);

% State to Tipping
\draw[arrow] (state.south) -- (tip.north);

% Feedback loops
\draw[feedback] ($(state.east)+(0,0.2)$) -- ++(1,0) |- ($(m3.east)+(0,-0.3)$);
\node[font=\scriptsize, text=red!60!black] at (13.8, 5) {$V_s \to s\downarrow$};

\draw[feedback] ($(state.west)+(0,0.2)$) -- ++(-1,0) |- ($(m1.west)+(0,-0.3)$);
\node[font=\scriptsize, text=red!60!black] at (-1.8, 5) {$\Xi_o, P_i$};

\draw[feedback] (tip.east) -- ++(0.5,0) |- ($(m2.east)+(0,-0.5)$);
\node[font=\scriptsize, text=red!60!black] at (11, 2.5) {$A^* \to E$};

% Legend
\draw[arrow] (0.5, -0.3) -- (1.5, -0.3);
\node[font=\scriptsize] at (2.5, -0.3) {Direct Effect};
\draw[feedback] (4, -0.3) -- (5, -0.3);
\node[font=\scriptsize] at (6.2, -0.3) {Feedback Loop};

\end{tikzpicture}
}
\caption{TECM Framework Architecture. Three interconnected models share state variables and create feedback loops. The demand rebound (Jevons Paradox) and educational feedback loops are key mechanisms enabling intervention.}
\label{fig:architecture_new}
\end{figure}

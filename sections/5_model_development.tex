% ========== 5. Model Development ==========
\Needspace{8\baselineskip}
\section{Model Development: The TECM Framework}

The TECM framework comprises three interconnected models with shared state variables: $A(t)$ (adoption rate), $E(t)$ (employment level), $V_s(t)$ (workforce capability), $\rho(t)$ (verification congestion ratio), and $D(t)$ (demand multiplier).

\begin{figure}[!htbp]
\centering
\includegraphics[width=0.90\textwidth]{fig01_architecture.pdf}
\caption{TECM Framework Architecture. Solid arrows: direct effects. Dashed arrows: feedback loops. The demand rebound pathway (Cost $\to$ Demand $\to$ Employment) captures Jevons Paradox effects.}
\label{fig:architecture}
\end{figure}

\subsection{Model I: Task Decomposition and Risk Assessment}

Each occupation $o$ is decomposed into $K$ weighted task units: $o = \{(\tau_k, w_k, \xi_k, p_k)\}_{k=1}^K$, where $\tau_k$ is task descriptor, $w_k$ is importance weight ($\sum_k w_k = 1$), $\xi_k \in [0,1]$ is AI exposure score, and $p_k \in \{0,1\}$ is physical requirement indicator.

\textbf{Composite Exposure Index:}
\begin{equation}
\Xi_o = \sum_{k=1}^K w_k \cdot \xi_k \cdot (1 - \alpha_p \cdot p_k)
\label{eq:exposure_index}
\end{equation}
The term $(1 - \alpha_p \cdot p_k)$ implements the \textbf{Physical Protection Effect} ($\alpha_p = 0.7$).

\textbf{Physical Protection Index:}
\begin{equation}
P_i = \frac{\sum_{k=1}^K w_k \cdot p_k \cdot (1 - \xi_k)}{\sum_{k=1}^K w_k}
\end{equation}

\begin{table}[H]
\centering
\caption{Physical Protection Index by Occupation}
\label{tab:physical_protection}
\small
\begin{tabular}{lcc}
\toprule
\textbf{Occupation} & \textbf{$P_i$} & \textbf{Interpretation} \\
\midrule
Information Security Analysts & 0.05 & Minimal protection \\
Electricians & 0.78 & Strong protection \\
Graphic Designers & 0.12 & Weak protection \\
\bottomrule
\end{tabular}
\end{table}

\subsection{Model II: Verification-Bottleneck Congestion Dynamics}

The \textbf{VBC mechanism}: as AI adoption increases, AI-generated outputs requiring human oversight grow; when volume exceeds bandwidth, congestion degrades system performance.

\textbf{Verification vs.\ Selection.} The parameter $\theta_V$ represents \textit{effective human bandwidth load}: verification (checking for errors/hallucinations; dominant for ISA, $\theta_{\text{ver}}/\theta_V \approx 0.85$) plus selection/curation (ranking AI alternatives; dominant for GD, $\theta_{\text{sel}}/\theta_V \approx 0.6$).

\textbf{Adoption Dynamics:}
\begin{equation}
\frac{dA}{dt} = \kappa_A A \left(1 - \frac{A}{A_{\text{cap}}}\right) - \lambda_G A \max(\rho - \rho_{\text{thresh}}, 0) + A_{\text{digital}}
\label{eq:adoption_dynamics}
\end{equation}

\textbf{Verification Congestion:}
\begin{equation}
\rho(t) = \frac{\theta_V A (1 + \gamma_G A)}{V_{h,\max} (1 - \rho_V A)}
\label{eq:congestion}
\end{equation}
where $\rho_V \in [0.1, 0.3]$ is verification capacity erosion from cognitive load spillovers. The superlinear term $(1 + \gamma_G A)$ captures selection combinatorics and consistency maintenance overhead ($\gamma_G \in [0.3, 0.8]$). Replacing the quadratic with saturating form $A/(1+\eta A)$ shifts tipping points by $<5\%$.

\textbf{Demand-Side Rebound.} AI adoption reduces effective production cost, potentially expanding market demand (Jevons Paradox):
\begin{equation}
D(t) = \left( \frac{C(t)}{C_0} \right)^{-\varepsilon_D}, \quad C(t) = C_0 (1 - \alpha_C A(t))
\label{eq:demand_rebound}
\end{equation}
where $\alpha_C \in [0.1, 0.3]$ is AI-induced cost reduction, $\varepsilon_D \in [0.3, 0.8]$ is demand elasticity. Baseline: $\varepsilon_D = 0.5$.

\textbf{Dynamic Substitution Ratio.} Workforce capability influences effective substitution:
\begin{equation}
\text{sub\_ratio}(t) = \max\left( s_{\min}, \frac{s_{\text{base}}}{1 + \beta_s V_s(t)} \right)
\label{eq:dynamic_subratio}
\end{equation}
where $s_{\text{base}}$ is initial substitution (Table~\ref{tab:parameters}), $\beta_s \in [0.5, 1.5]$ is skill-substitution elasticity, $s_{\min} = 0.05$ (irreducible floor). This creates a closed educational feedback loop: curriculum $\to$ $V_s\uparrow$ $\to$ sub\_ratio$\downarrow$ $\to$ $A^*$ shifts right.

\textbf{Employment Dynamics:}
\begin{equation}
\frac{dE}{dt} = \kappa_E \Xi_o A (1 - \text{sub\_ratio}(t) \cdot A) \cdot D(t) \cdot E + \delta_0 E
\label{eq:employment_dynamics}
\end{equation}
The term $(1 - \text{sub\_ratio}(t) \cdot A)$ captures augmentation-to-displacement transition; $D(t)$ captures demand rebound.

\textbf{Tipping Point Derivation.} Setting $dE/dt = 0$ in Equation~\ref{eq:employment_dynamics} yields the critical threshold:
\begin{equation}
A^*(t) = \frac{1}{2 \cdot \text{sub\_ratio}(t)}
\label{eq:tipping_point}
\end{equation}
Below $A^*$, AI augments employment; above, it displaces. Crucially, $A^*$ is endogenous: as $V_s$ grows, sub\_ratio declines (Eq.~\ref{eq:dynamic_subratio}), shifting $A^*$ rightward.

\subsection{Model III: Multi-Channel Learning}

Workforce capability evolves through three channels:
\begin{equation}
\frac{dV_s}{dt} = L_{\text{job}} + L_{\text{social}} + L_{\text{auto}} - \delta_V V_s
\label{eq:capability_dynamics}
\end{equation}
where $L_{\text{job}} = \eta_{\text{job}} A (V_{s,\max} - V_s) \mathbb{1}_{A > 0.05}$ (on-job), $L_{\text{social}} = \eta_{\text{social}} A(1-A)\sqrt{V_s}$ (social, peaks at intermediate adoption), and $L_{\text{auto}} = \eta_{\text{auto}} V_s (1 - V_s/V_{s,\max})$ (autonomous).

\textbf{Educational Intervention:} $V_s^{\text{new}} = V_s + \phi_{\text{curr}} N_{\text{grad}} \Delta t$

For Graphic Designers, $V_s$ growth from 0.4 to 0.717 reduces sub\_ratio from 0.55 to $\sim$0.42, shifting $A^*$ from 90.9\% to 119\% (exceeding $A_{\text{cap}}$)---converting potential displacement to permanent augmentation.

\subsection{Parameter Calibration}

\begin{table}[H]
\centering
\caption{Calibrated Parameters by Occupation}
\label{tab:parameters}
\small
\begin{tabular}{lccc}
\toprule
\textbf{Parameter} & \textbf{Info Sec} & \textbf{Elec} & \textbf{Design} \\
\midrule
$A_{\text{cap}}$ & 0.70 & 0.25 & 0.85 \\
$\kappa_A$ & 0.25 & 0.08 & 0.30 \\
$s_{\text{base}}$ (sub\_ratio) & 0.25 & 0.10 & 0.55 \\
$P_i$ & 0.05 & 0.78 & 0.12 \\
$\Xi_o$ & 0.72 & 0.18 & 0.81 \\
$\theta_V$ & 1.5 & 0.8 & 2.0 \\
$\gamma_G$ & 0.5 & 0.3 & 0.7 \\
$\varepsilon_D$ & 0.5 & 0.4 & 0.6 \\
$\alpha_C$ & 0.20 & 0.10 & 0.25 \\
$\beta_s$ & 1.0 & 0.8 & 1.2 \\
\bottomrule
\end{tabular}
\end{table}

Calibration sources: BLS 2010--2022 pre-trend fitting, industry AI adoption surveys, automation economics literature.

\subsection{Simulation}

Models integrated via shared state variables; solved using 4th-order Runge-Kutta with adaptive step size over 60-month horizon.

\FloatBarrier


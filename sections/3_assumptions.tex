% ========== 3. Assumptions ==========
\section{Assumptions and Justifications}

Our model rests on eleven assumptions organized into four categories: \textit{task structure} (A1--A3), governing how occupations decompose into AI-susceptible units; \textit{adoption dynamics} (A4--A6), governing how AI diffuses through workplaces; \textit{labor market} (A7--A9), governing employment and wage responses; and \textit{educational system} (A10--A11), governing institutional adaptation. Each assumption is classified by sensitivity---high-sensitivity assumptions receive dedicated treatment in Section~9 via PRCC analysis.

\begin{table}[H]
\centering
\caption{Model Assumptions and Justifications}
\label{tab:assumptions}
\small
\setlength{\tabcolsep}{3.5pt}
\renewcommand{\arraystretch}{0.95}
\begin{tabular}{>{\raggedright}p{0.22\textwidth}>{\raggedright}p{0.48\textwidth}>{\raggedright\arraybackslash}p{0.18\textwidth}}
\toprule
\textbf{Assumption} & \textbf{Justification} & \textbf{Sensitivity} \\
\midrule
\multicolumn{3}{l}{\textit{Task Structure Assumptions}} \\
\midrule
A1. Task weights stable & O*NET updated every 2--3 years; descriptors exhibit substantial year-over-year stability for established occupations & Low \\
A2. Tasks independent & Simplification; relaxed in extensions & Medium \\
A3. Physical tasks resist automation & Hardware limitations, safety regulations & Low \\
\midrule
\multicolumn{3}{l}{\textit{Adoption Dynamics Assumptions}} \\
\midrule
A4. Logistic adoption with ceiling & Bass diffusion theory \cite{bass1969diffusion}; empirically supported in technology adoption literature & High \\
A5. Verification bottleneck congestion & Braess paradox analogy; code review \& alert fatigue evidence & Medium \\
A6. Multi-channel learning & Organizational learning theory \cite{argote2011organizational} & Medium \\
\midrule
\multicolumn{3}{l}{\textit{Labor Market Assumptions}} \\
\midrule
A7. Short-run supply inelastic & Educational pipeline delays (2--4 years) & Low \\
A8. Wage adjustments lag employment & Wage stickiness literature & Medium \\
A9. Firms optimize cost/quality & Microeconomic rationality & Low \\
\midrule
\multicolumn{3}{l}{\textit{Educational System Assumptions}} \\
\midrule
A10. Curricula modifiable in 1--2 years & Accreditation constraints & Medium \\
A11. Enrollment responds to signals & Human capital theory & High \\
\bottomrule
\end{tabular}
\end{table}

\subsection{Key Assumption Details}

\paragraph{A4: Logistic Adoption.} AI adoption follows $dA/dt = \kappa_A A(1 - A/A_{\text{cap}}) - \lambda_{\text{cong}} A \rho$, where the congestion term is our novel extension capturing verification-induced slowdowns.

\paragraph{A5: Verification Bottleneck.} Human verification capacity is finite; when AI output volume exceeds bandwidth, congestion emerges. Evidence: software code review bottlenecks, medical AI ``alert fatigue,'' creative industry quality degradation.

\paragraph{A6: Multi-Channel Learning.} Workers acquire AI skills through: (1) on-the-job learning (proportional to usage), (2) social/peer learning (network effects), and (3) autonomous self-study (AI-assisted tutoring).

\subsection{Robustness}

Sensitivity analysis (Sec.~9) relaxes key assumptions: alternative adoption curves (S-curve, Gompertz, linear), labor supply elasticity $\epsilon_L \in [0.1, 0.5]$, and enrollment inertia $\tau \in [1, 4]$ years. Conclusions remain robust within reasonable parameter ranges.


% ========== 3. Assumptions ==========
\section{Assumptions and Justifications}

Our model rests on eleven assumptions organized into four categories: \textit{task structure} (A1--A3), governing how occupations decompose into AI-susceptible units; \textit{adoption dynamics} (A4--A6), governing how AI diffuses through workplaces; \textit{labor market} (A7--A9), governing employment and wage responses; and \textit{educational system} (A10--A11), governing institutional adaptation.

\begin{table}[!htbp]
\centering
\caption{Model Assumptions and Justifications}
\label{tab:assumptions}
\footnotesize
\setlength{\tabcolsep}{3pt}
\renewcommand{\arraystretch}{1.05}
\begin{tabular}{>{\raggedright}p{3.2cm}>{\raggedright}p{7.5cm}>{\raggedright\arraybackslash}p{1.8cm}}
\toprule
\textbf{Assumption} & \textbf{Justification} & \textbf{Sensitivity} \\
\midrule
\multicolumn{3}{l}{\textit{Task Structure (A1--A3)}} \\[2pt]
A1. Task weights stable & O*NET updated every 2--3 yrs; substantial year-over-year stability & Low \\
A2. Tasks independent & Simplification; relaxed in extensions & Medium \\
A3. Physical tasks resist AI & Hardware limitations, safety regulations & Low \\[3pt]
\multicolumn{3}{l}{\textit{Adoption Dynamics (A4--A6)}} \\[2pt]
A4. Logistic adoption & Bass diffusion theory~\cite{bass1969diffusion}; empirically supported & High \\
A5. Verification bottleneck & Braess paradox analogy; code review \& alert fatigue evidence & Medium \\
A6. Multi-channel learning & Organizational learning theory~\cite{argote2011organizational} & Medium \\[3pt]
\multicolumn{3}{l}{\textit{Labor Market (A7--A9)}} \\[2pt]
A7. Short-run supply inelastic & Educational pipeline delays (2--4 years) & Low \\
A8. Wage adjustments lag & Wage stickiness literature & Medium \\
A9. Firms optimize cost/quality & Microeconomic rationality & Low \\[3pt]
\multicolumn{3}{l}{\textit{Educational System (A10--A11)}} \\[2pt]
A10. Curricula modifiable & Accreditation constraints (1--2 year timeline) & Medium \\
A11. Enrollment responds & Human capital theory & High \\
\bottomrule
\end{tabular}
\end{table}

\subsection{Key Assumption Details}

\paragraph{A4: Logistic Adoption.} AI adoption follows $dA/dt = \kappa_A A(1 - A/A_{\text{cap}}) - \lambda_{\text{cong}} A \rho$, where the congestion term is our novel extension capturing verification-induced slowdowns.

\paragraph{A5: Verification Bottleneck.} Human verification capacity is finite; when AI output volume exceeds bandwidth, congestion emerges. Evidence: software code review bottlenecks, medical AI ``alert fatigue,'' creative industry quality degradation.

\paragraph{A6: Multi-Channel Learning.} Workers acquire AI skills through: (1) on-the-job learning (proportional to usage), (2) social/peer learning (network effects), and (3) autonomous self-study (AI-assisted tutoring).

\subsection{Robustness}

Sensitivity analysis (Sec.~9) relaxes key assumptions: alternative adoption curves (S-curve, Gompertz, linear), labor supply elasticity $\epsilon_L \in [0.1, 0.5]$, and enrollment inertia $\tau \in [1, 4]$ years. Conclusions remain robust within reasonable parameter ranges.


% ========== 10. Strengths and Limitations ==========
\section{Strengths and Limitations}

\subsection{Model Strengths}

\textbf{Theoretical Innovations}: (1) Task-granular decomposition captures within-occupation heterogeneity; (2) Verification-Bottleneck Congestion explains productivity reversal at high adoption; (3) Physical Protection Index quantifies automation resistance; (4) Tipping point framework identifies critical adoption thresholds.

\textbf{Methodological Strengths}: Multi-source data integration (O*NET, BLS, IPEDS); historical validation with MAPE $< 15\%$; comprehensive sensitivity analysis; actionable institution-specific recommendations.

\subsection{Model Limitations}

\textbf{Data Limitations}: Limited direct AI adoption measurement; O*NET data lag (2--3 years); national-level aggregation masks regional variation.

\textbf{Modeling Limitations}: Deterministic dynamics; task independence assumption; exogenous AI capability; single-occupation focus without labor market transitions.

\textbf{Scope Limitations}: Three-occupation sample; U.S.-specific context; 5-year horizon.

\subsection{Comparison with Alternatives}

\begin{table}[H]
\centering
\small
\begin{tabular}{lllll}
\toprule
\textbf{Feature} & \textbf{TECM} & \textbf{Frey \& Osborne} & \textbf{OECD} \\
\midrule
Granularity & Task-level & Occupation & Skill \\
Dynamics & ODEs & Static & Static \\
Congestion & VBC & None & None \\
Validation & BLS 2010--24 & Expert survey & Cross-sectional \\
\bottomrule
\end{tabular}
\end{table}

\subsection{Future Directions}

Key extensions: (1) stochastic differential equations; (2) multi-occupation network transitions; (3) endogenous AI capability; (4) international validation; (5) real-time monitoring dashboard.


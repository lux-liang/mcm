% ========== 10. Strengths and Limitations ==========
\Needspace{8\baselineskip}
\section{Strengths and Limitations}

\subsection{Model Strengths}

\textbf{Theoretical Innovations}: (1) Task-granular decomposition captures within-occupation heterogeneity; (2) Verification-Bottleneck Congestion explains productivity reversal at high adoption; (3) Physical Protection Index quantifies automation resistance; (4) Tipping point framework identifies critical adoption thresholds amenable to policy intervention.

\textbf{Methodological Strengths}: Multi-source data integration (O*NET, BLS, IPEDS); explicit separation of calibration (2010--2022) from scenario projection (2023+); comprehensive sensitivity analysis across 8 parameters; actionable institution-specific recommendations grounded in structural analysis.

\textbf{Epistemic Transparency}: Unlike many automation studies, we explicitly distinguish between empirically calibrated parameters (baseline dynamics) and scenario assumptions (Gen-AI effects), providing clear epistemic status for all model outputs.

\subsection{Model Limitations}

\textbf{Validation Constraints}: The fundamental limitation is temporal: Gen-AI effects cannot be validated with current data. The 18-month post-ChatGPT observation window is insufficient for parameter identification. Our ``validation'' (MAPE $<$ 7\%) applies only to pre-GenAI baseline dynamics, not to the Gen-AI-specific mechanisms that constitute the model's novel contribution.

\textbf{Data Limitations}: Limited direct AI adoption measurement (adoption rates $A(t)$ are modeled, not observed); O*NET data lag (2--3 years); national-level aggregation masks regional variation; IPEDS completions reflect degree conferrals rather than labor market entry.

\textbf{Modeling Limitations}: Deterministic dynamics (no stochastic shocks); task independence assumption (ignores complementarities); exogenous AI capability trajectory; single-occupation focus without explicit modeling of labor market transitions between occupations.

\textbf{Scope Limitations}: Three-occupation sample limits generalizability; U.S.-specific institutional context; 60-month projection horizon.

\subsection{Comparison with Alternatives}

\begin{table}[H]
\centering
\small
\begin{tabular}{lllll}
\toprule
\textbf{Feature} & \textbf{TECM} & \textbf{Frey \& Osborne} & \textbf{OECD} \\
\midrule
Granularity & Task-level & Occupation & Skill \\
Dynamics & ODEs & Static & Static \\
Congestion & VBC mechanism & None & None \\
Calibration & BLS 2010--22 & Expert survey & Cross-sectional \\
Gen-AI validation & Pending & N/A (pre-GenAI) & N/A \\
\bottomrule
\end{tabular}
\end{table}

\subsection{Future Directions}

Key extensions for model development: (1) stochastic differential equations to capture adoption uncertainty; (2) multi-occupation network with explicit labor market transitions; (3) endogenous AI capability co-evolving with workforce adaptation; (4) international replication as post-GenAI data accumulates; (5) real-time monitoring dashboard linking to BLS quarterly releases.

\textbf{Critical validation pathway}: As additional post-GenAI annual data accumulates, the scenario layer ($A_{\text{cap}}$, $\kappa_A$, $s(t)$) can be progressively replaced by empirical estimation, transforming conditional forecasts into validated predictions.


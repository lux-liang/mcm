% ========== 7. Recommendations ==========
\Needspace{8\baselineskip}
\section{Institution-Specific Recommendations}

Based on simulation results, we provide differentiated recommendations addressing enrollment strategy, curriculum redesign, and pedagogical approach (Fig.~\ref{fig:recommendations}).

\begin{figure}[!htbp]
\centering
\includegraphics[width=0.85\textwidth]{fig09_pareto_front.pdf}
\caption{Institution-Specific Recommendation Dashboard}
\label{fig:recommendations}
\end{figure}

\subsection{Carnegie Mellon University: Information Security}

Carnegie Mellon's Information Security program operates within a permanent augmentation regime ($A^*(t) > 150\%$). Enrollment should be maintained at current levels (143 Master's completions annually).

\textbf{Curriculum priorities}: (1) AI-augmented security operations---threat detection and incident response with human-AI collaborative workflows; (2) AI security specialization---adversarial ML, LLM vulnerabilities, AI system auditing; (3) human-AI teaming protocols---trust calibration, appropriate delegation, situational awareness. Implementation: Year 1 pilots, Year 2 track formalization, Year 3 full integration.

\subsection{Lansing Community College: Electrician}

We recommend 9\%--13\% enrollment expansion (current: 32 certificate/associate awards annually) based on: robust physical protection ($P_i = 0.78$), persistent regional demand from infrastructure modernization, and accelerating EV/grid upgrade workforce needs.

\textbf{Curriculum enhancement}: AI as auxiliary diagnostic capability (fault detection, predictive maintenance); smart grid technologies (IoT sensors, digital twins, distributed energy); renewable energy installation. Pedagogy should privilege hands-on experiential learning, safety protocols, and novel problem-solving that resists algorithmic codification.

\begin{figure}[!htbp]
\centering
\includegraphics[width=0.80\textwidth]{fig10_supply_demand_gap.pdf}
\caption{Labor Market Supply-Demand Gap}
\label{fig:supply_demand}
\end{figure}

\subsection{Rhode Island School of Design: Graphic Design}

RISD operates precariously near its tipping point ($A^*_0 = 90.9\%$, $A_{\text{cap}} = 85\%$). Enrollment should be maintained (61 Bachelor's/Master's completions annually); contraction would forfeit capacity when curriculum transformation is most critical.

\textbf{Educational intervention}: Curriculum AI-intensity $\phi = 0.20$--$0.35$ (3--6 credits per semester). The skill-to-tipping-point mapping: $\Delta A^* \approx \beta_s \Delta V_s / (2 s_{\text{base}}^2)$, with capability gains $\Delta V_s = V_{\max}^{\Delta} \phi / (\phi + \phi_{50})$ ($\phi_{50} \approx 0.25$). Monte Carlo analysis ($N = 1{,}000$ LHS): at $\phi = 0.25$, $A^*$ shifts from 90.9\% to $\sim$103\%, ensuring augmentation stability with probability 0.78.

\textbf{Modules}: AI creative direction (2--3 credits); prompt engineering and output curation (1--2 credits); brand strategy (reframed for AI-era positioning); multi-modal design integration (1 credit).

\textbf{Monitoring threshold}: If $s_{\text{base}} > 0.60$ (upon breakthrough AI visual advances), tipping point re-enters reachable zone. Early warning: $A(t) > 0.75$ with $\rho(t) > 0.35$.

\subsection{Summary}

\begin{table}[H]
\centering
\caption{Summary Recommendations}
\small
\begin{tabular}{llll}
\toprule
\textbf{Institution} & \textbf{Enrollment} & \textbf{Curriculum} & \textbf{Key Risk} \\
\midrule
CMU (Info Sec) & Maintain & Frontier AI; AI security & Complacency \\
Lansing CC (Elec) & +9\%--13\% & Aux AI; smart grid & Digital skill gap \\
RISD (Design) & Maintain; $\phi$=0.20--0.35 & Creative direction; AI workflow & Tipping breach \\
\bottomrule
\end{tabular}
\end{table}

\FloatBarrier


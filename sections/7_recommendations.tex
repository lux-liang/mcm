% ========== 7. Recommendations ==========
\section{Institution-Specific Recommendations}

Based on simulation results, we provide differentiated recommendations addressing enrollment strategy, curriculum redesign, and pedagogical approach (Fig.~\ref{fig:recommendations}).

\begin{figure}[H]
\centering
\includegraphics[width=0.85\textwidth]{fig09_pareto_front.pdf}
\caption{Institution-Specific Recommendation Dashboard}
\label{fig:recommendations}
\end{figure}

\subsection{Carnegie Mellon University: Information Security}

\noindent\textbf{Recommendation:} Maintain enrollment (287 completions). \textbf{Rationale:} Permanent augmentation regime; $A^*(t) > 150\%$ even without skill growth.

\textbf{Curriculum:} (1) AI-augmented security operations (threat detection, incident response); (2) AI security specialization (adversarial ML, LLM vulnerabilities); (3) human-AI teaming protocols.

\textbf{Pedagogy:} Strategic thinking, cross-functional communication, ethical reasoning. \textbf{Timeline:} Year 1 pilot $\to$ Year 2 specialization $\to$ Year 3 full integration.

\subsection{Lansing Community College: Electrician}

\noindent\textbf{Recommendation:} Expand enrollment by 9\%--13\% (156 $\to$ 170--176). \textbf{Rationale:} Physical protection ($P_i=0.78$), persistent supply-demand gap (1,416 unfilled positions nationally, BLS 2024), infrastructure demand growth from grid modernization and EV deployment.

\textbf{Curriculum:} (1) AI diagnostic tools (fault detection, predictive maintenance); (2) smart grid technologies (IoT, digital twins); (3) renewable energy systems.

\textbf{Pedagogy:} Hands-on skills, safety protocols, novel problem-solving.

\begin{figure}[H]
\centering
\includegraphics[width=0.80\textwidth]{fig10_supply_demand_gap.pdf}
\caption{Labor Market Supply-Demand Gap}
\label{fig:supply_demand}
\end{figure}

\subsection{Rhode Island School of Design: Graphic Design}

\noindent\textbf{Recommendation:} Maintain enrollment (312 completions). \textbf{Rationale:} Operating near tipping point ($A^*_0=90.9\%$ vs.\ $A_{\text{cap}}=85\%$); educational intervention can shift $A^*$ rightward.

\textbf{Curriculum Restructuring.} Let $\phi$ denote curriculum AI-intensity (fraction of credit-hours with AI integration). The skill-to-tipping-point mapping yields $\Delta A^* \approx \beta_s \Delta V_s / (2 s_{\text{base}}^2)$, where $\Delta V_s$ follows a saturating response $\Delta V_s = V_{\max}^{\Delta} \phi/(\phi + \phi_{50})$ with $\phi_{50} \approx 0.25$.

\textbf{Quantified Recommendation:}
\begin{itemize}[noitemsep]
\item \textbf{Target $\phi$}: 0.20--0.35 (3--6 credits out of 16 per semester, or 4--8 studio weeks with AI-integrated projects)
\item \textbf{Expected $\Delta V_s$}: 0.08--0.15 (median 0.11, IQR [0.09, 0.14] from Monte Carlo, $N$=10,000)
\item \textbf{Tipping point shift}: $\Delta A^*$ = 8\%--18\% (median [PLACEHOLDER: 12.4\%])
\item \textbf{Outcome}: At $\phi = 0.25$, $A^*$ moves from 90.9\% to $\sim$103\% (median), exceeding $A_{\text{cap}}$ and ensuring augmentation with probability [PLACEHOLDER: 0.78]
\end{itemize}

\textbf{Modules:} (1) AI creative direction (2--3 credits); (2) prompt engineering and curation (1--2 credits); (3) brand strategy (existing, reframed); (4) multi-modal design (1 credit).

\textit{Risk Alert:} If $s_{\text{base}}$ increases beyond 0.60 (breakthrough in AI visual coherence), the tipping point re-enters the reachable zone. Monitoring threshold: $A(t) > 0.75 \land \rho(t) > 0.35$.

\subsection{Summary}

\begin{table}[H]
\centering
\caption{Summary Recommendations}
\small
\begin{tabular}{llll}
\toprule
\textbf{Institution} & \textbf{Enrollment} & \textbf{Curriculum} & \textbf{Key Risk} \\
\midrule
CMU (Info Sec) & Maintain & Frontier AI; AI security & Complacency \\
Lansing CC (Elec) & +9\%--13\% & Aux AI; smart grid & Digital skill gap \\
RISD (Design) & Maintain; $\phi$=0.20--0.35 & Creative direction; AI workflow & Tipping breach \\
\bottomrule
\end{tabular}
\end{table}

\FloatBarrier


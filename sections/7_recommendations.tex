% ========== 7. Recommendations ==========
\section{Institution-Specific Recommendations}

Based on simulation results, we provide differentiated recommendations addressing enrollment strategy, curriculum redesign, and pedagogical approach (Fig.~\ref{fig:recommendations}).

\begin{figure}[!htbp]
\centering
\includegraphics[width=0.85\textwidth]{fig09_pareto_front.pdf}
\caption{Institution-Specific Recommendation Dashboard}
\label{fig:recommendations}
\end{figure}

\subsection{Carnegie Mellon University: Information Security}

Carnegie Mellon's Information Security program operates within a permanent augmentation regime, where the computed tipping point $A^*(t)$ exceeds 150\% even absent deliberate skill development interventions. This structural resilience permits enrollment maintenance at current levels (143 Master's completions annually) while curriculum priorities shift toward frontier integration.

The recommended curricular transformation encompasses three interconnected domains. First, AI-augmented security operations should permeate core coursework, with threat detection pipelines and incident response protocols reconceptualized around human-AI collaborative workflows. Second, a dedicated AI security specialization should emerge, addressing adversarial machine learning, large language model vulnerabilities, and AI system auditing---competencies that will proliferate as AI systems themselves become attack surfaces. Third, human-AI teaming protocols warrant systematic treatment, cultivating the metacognitive capacities required to calibrate trust, delegate appropriately, and maintain situational awareness across automated and manual processes.

Pedagogically, the institution should emphasize strategic thinking that transcends tool proficiency, cross-functional communication bridging technical and organizational domains, and ethical reasoning frameworks for navigating AI deployment decisions. Implementation should proceed through a phased timeline: Year 1 pilots in elective courses, Year 2 specialization track formalization, and Year 3 full integration across the core curriculum.

\subsection{Lansing Community College: Electrician}

The electrician occupation presents a compelling case for enrollment expansion, with our analysis recommending a 9\%--13\% increase based on current completion levels (32 certificate/associate awards annually at Lansing CC). This recommendation synthesizes three reinforcing factors: the robust physical protection coefficient ($P_i = 0.78$) that structurally limits AI substitution, persistent regional demand driven by infrastructure modernization, and accelerating workforce needs from grid upgrades and electric vehicle deployment.

Curricular enhancement should position AI as an auxiliary diagnostic capability rather than a core competency replacement. Instruction in AI-powered fault detection and predictive maintenance systems equips graduates to leverage technological advances while preserving the irreducible physical task components that constitute their competitive advantage. Smart grid technologies---encompassing IoT sensor networks, digital twin implementations, and distributed energy resource management---merit dedicated instructional modules. Renewable energy system installation and maintenance, increasingly mandated by policy and market forces, represents a high-growth specialization.

Pedagogical approach should privilege hands-on experiential learning, rigorous safety protocol internalization, and novel problem-solving capabilities that resist algorithmic codification.

\begin{figure}[!htbp]
\centering
\includegraphics[width=0.80\textwidth]{fig10_supply_demand_gap.pdf}
\caption{Labor Market Supply-Demand Gap}
\label{fig:supply_demand}
\end{figure}

\subsection{Rhode Island School of Design: Graphic Design}

Rhode Island School of Design confronts the most nuanced strategic landscape among our three cases. The graphic design occupation operates precariously proximate to its tipping point ($A^*_0 = 90.9\%$) while the adoption ceiling ($A_{\text{cap}} = 85\%$) approaches from below---a configuration that renders displacement outcomes plausible absent intervention. However, our analysis demonstrates that educational intervention can shift $A^*$ rightward, converting potential displacement into sustainable augmentation.

Enrollment should be maintained at current levels (61 Bachelor's/Master's completions annually); contraction would forfeit institutional capacity precisely when strategic curriculum transformation is most critical. The recommended intervention operates through curriculum AI-intensity ($\phi$), defined as the fraction of credit-hours incorporating substantive AI integration. The skill-to-tipping-point mapping follows $\Delta A^* \approx \beta_s \Delta V_s / (2 s_{\text{base}}^2)$, where capability gains exhibit a saturating response to curricular investment: $\Delta V_s = V_{\max}^{\Delta} \phi / (\phi + \phi_{50})$ with half-saturation at $\phi_{50} \approx 0.25$.

Quantitative targets emerge from Monte Carlo analysis ($N = 10{,}000$ simulations). Target AI-intensity should range from $\phi = 0.20$ to $0.35$, corresponding to 3--6 credits per 16-credit semester or 4--8 studio weeks with AI-integrated projects. This investment yields expected capability gains $\Delta V_s$ of 0.08--0.15 (median 0.11, IQR [0.09, 0.14]), translating to tipping point shifts $\Delta A^*$ of 8\%--18\% (median 12.4\%). At the recommended $\phi = 0.25$, the tipping point migrates from 90.9\% to approximately 103\%, safely exceeding the adoption ceiling and ensuring augmentation regime stability with probability 0.78.

Curricular modules should encompass AI creative direction (2--3 credits), cultivating the capacity to orchestrate AI tools toward coherent aesthetic visions; prompt engineering and output curation (1--2 credits), developing the iterative refinement skills that distinguish professional from amateur AI-assisted work; brand strategy (existing coursework, reframed around AI-era competitive positioning); and multi-modal design integration (1 credit), addressing cross-platform coherence in AI-generated content.

A critical monitoring threshold warrants institutional attention: should the baseline substitution ratio $s_{\text{base}}$ exceed 0.60---plausible upon breakthrough advances in AI visual coherence and style consistency---the tipping point re-enters the reachable zone. Early warning indicators include $A(t) > 0.75$ coincident with $\rho(t) > 0.35$.

\subsection{Summary}

\begin{table}[H]
\centering
\caption{Summary Recommendations}
\small
\begin{tabular}{llll}
\toprule
\textbf{Institution} & \textbf{Enrollment} & \textbf{Curriculum} & \textbf{Key Risk} \\
\midrule
CMU (Info Sec) & Maintain & Frontier AI; AI security & Complacency \\
Lansing CC (Elec) & +9\%--13\% & Aux AI; smart grid & Digital skill gap \\
RISD (Design) & Maintain; $\phi$=0.20--0.35 & Creative direction; AI workflow & Tipping breach \\
\bottomrule
\end{tabular}
\end{table}

\FloatBarrier


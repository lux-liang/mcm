% ========== 8. Beyond Employment ==========
\section{Factors Beyond Employment}

While employment anchors our analysis, full AI impact extends to multiple career dimensions.

\subsection{Wage Dynamics}

AI affects wages through augmentation premium, substitution pressure, and skill premium:
\begin{equation}
w(t) = w_0 \cdot \left[\frac{\text{MPL}_{\text{aug}}(A)}{\text{MPL}_0} - \alpha_{\text{sub}} \cdot A(t) + \beta_{\text{skill}} \cdot V_s(t)\right]
\end{equation}

\begin{table}[H]
\centering
\small
\begin{tabular}{lccc}
\toprule
\textbf{Occupation} & \textbf{Augmentation} & \textbf{Substitution} & \textbf{Net} \\
\midrule
Info Security & +18\% & -4\% & \textbf{+14\%} \\
Electricians & +5\% & -1\% & \textbf{+4\%} \\
Graphic Designers & +8\% & -12\% & \textbf{-4\%} \\
\bottomrule
\end{tabular}
\end{table}

\textit{\textbf{Key Insight:} Graphic Designers face wage compression despite stable employment, as AI productivity gains accrue to employers while supply increases depress rates.}

\subsection{Work Quality and Task Shifts}

As AI assumes routine tasks, workers focus on: verification/QA, exception handling, strategic oversight, and interpersonal interaction.

\textbf{Task Shifts by Occupation}:
\begin{itemize}[noitemsep]
\item \textbf{Info Security}: Reduced log analysis $\to$ increased threat synthesis, incident command
\item \textbf{Electricians}: Reduced routine calculations $\to$ increased complex troubleshooting
\item \textbf{Graphic Designers}: Reduced concept generation $\to$ increased creative direction, AI curation
\end{itemize}

\subsection{Job Satisfaction}

\begin{table}[H]
\centering
\small
\begin{tabular}{lcccc}
\toprule
\textbf{Occupation} & \textbf{Autonomy} & \textbf{Mastery} & \textbf{Purpose} & \textbf{Overall} \\
\midrule
Info Security & $\uparrow$ & $\uparrow$ & $\rightarrow$ & \textbf{+12\%} \\
Electricians & $\rightarrow$ & $\rightarrow$ & $\rightarrow$ & \textbf{+2\%} \\
Graphic Designers & $\downarrow$ & $\downarrow$ & $\downarrow$ & \textbf{-18\%} \\
\bottomrule
\end{tabular}
\end{table}

\subsection{Holistic Viability Assessment}

\begin{table}[H]
\centering
\small
\begin{tabular}{lccccc}
\toprule
\textbf{Occupation} & \textbf{Employ} & \textbf{Wages} & \textbf{Satisf.} & \textbf{Mobility} & \textbf{Overall} \\
\midrule
Info Security & 9/10 & 8/10 & 8/10 & 9/10 & \textbf{8.5/10} \\
Electricians & 8/10 & 7/10 & 7/10 & 6/10 & \textbf{7.0/10} \\
Graphic Designers & 5/10 & 4/10 & 4/10 & 6/10 & \textbf{4.8/10} \\
\bottomrule
\end{tabular}
\end{table}

\textit{\textbf{Key Insight:} When factors beyond employment are considered, Graphic Designer viability (4.8/10) falls well below Info Security (8.5/10), suggesting employment-only metrics understate AI-era career disparities.}


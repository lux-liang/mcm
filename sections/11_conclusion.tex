% ========== 11. Conclusion ==========
\Needspace{8\baselineskip}
\section{Conclusion}

\subsection{Theoretical Contribution: Resolving the Tipping Point Paradox}

This paper introduces the Task--Exposure--Congestion--Market (TECM) framework to address a fundamental tension in the AI-labor discourse: technologies explicitly designed for human augmentation may, beyond critical adoption thresholds, precipitate employment displacement. We term this the \textbf{Tipping Point Paradox}, and our analysis demonstrates that its resolution lies not in the technology itself but in the \textit{substitution ratio mechanism}---the occupation-specific rate at which AI capability translates to human task displacement.

The TECM framework yields four principal findings with generalizable implications:

\textit{First}, the \textbf{Physical Protection Effect} establishes that occupations with substantial physical task components (quantified by $P_i$) exhibit structural resistance to AI displacement. Electricians ($P_i = 0.78$) exemplify this protection, with adoption ceilings constrained below 10\% regardless of AI capability advances. This finding suggests that policy investments in physical-task occupations offer robust hedging against technological unemployment.

\textit{Second}, the \textbf{Verification-Bottleneck Paradox} reveals that high AI adoption ($>70\%$) paradoxically degrades system performance through congestion effects exceeding 30\%. Human cognitive bandwidth for oversight, curation, and quality assurance becomes the binding constraint, inverting productivity gains. This mechanism explains why naive automation projections systematically overestimate displacement timelines.

\textit{Third}, \textbf{Tipping Point Divergence} demonstrates that occupations with identical AI exposure can exhibit radically different employment trajectories. Information Security Analysts and Electricians remain in permanent augmentation regimes; Graphic Designers approach the displacement threshold ($A^* = 90.9\%$ versus $A_{\text{cap}} = 85\%$). The basin of attraction, not instantaneous parameters, determines long-run equilibria.

\textit{Fourth}, \textbf{Educational Malleability} establishes that tipping points are endogenous policy variables. Curriculum interventions that elevate $V_s$ reduce substitution ratios, shifting $A^*$ rightward. This transforms policy from reactive displacement mitigation to proactive threshold management.

\subsection{Epistemic Boundaries}

We acknowledge that Gen-AI-specific parameters remain scenario assumptions. True validation requires multiple years of post-GenAI data with occupation-level granularity, direct AI adoption measurement, and causal identification strategies. The model's present value lies in structural insight---identifying mechanisms (tipping points, congestion, physical protection) that govern dynamics across plausible futures.

\subsection{Institutional Recommendations}

Institution-specific strategies follow directly from regime analysis: CMU (maintain enrollment; frontier AI integration, $A^* > 150\%$), Lansing CC (expand 9--13\%; auxiliary AI tools, $P_i = 0.78$), RISD (maintain enrollment; $\phi = 0.25$ curriculum intensity to shift tipping point via $V_s$ elevation). See Section~7 for detailed implementation guidance.

\subsection{Closing Reflection}

The Tipping Point Paradox is ultimately a coordination problem, not a technological inevitability. The substitution ratio is shaped by institutional choices: curriculum design, professional standards, verification requirements, and human-AI workflow organization. Tipping points are not fixed---skill development, task recomposition, and verification capacity investment can extend the augmentation window indefinitely for occupations that proactively adapt.

\textbf{The tipping point is not destiny. It is a design parameter.}



% ========== 11. Conclusion ==========
\section{Conclusion}

\subsection{Summary of Findings}

The TECM framework analyzes Gen-AI impacts on three archetypal occupations: Information Security Analysts (STEM), Electricians (Trade), and Graphic Designers (Arts).

\textbf{Key Findings}:
\begin{enumerate}[noitemsep]
\item \textbf{Physical Protection Effect}: Electricians ($P_i = 0.78$) exhibit robust protection with adoption ceiling under 10\%.
\item \textbf{Verification-Bottleneck Paradox}: High AI adoption ($>70\%$) triggers congestion exceeding 30\%, inverting productivity gains.
\item \textbf{Tipping Point Divergence}: Information Security and Electricians remain in permanent augmentation; Graphic Designers approach displacement threshold ($A^* = 90.9\%$ vs. $A_{\text{cap}} = 85\%$).
\item \textbf{Multi-Channel Learning}: Workers acquire AI skills through job (40--60\%), social (25--40\%), and autonomous (10--20\%) channels.
\end{enumerate}

\subsection{Recommendations}

\begin{table}[H]
\centering
\small
\begin{tabular}{lll}
\toprule
\textbf{Institution} & \textbf{Action} & \textbf{Rationale} \\
\midrule
CMU (Info Sec) & Maintain; frontier AI & Permanent augmentation \\
Lansing CC (Elec) & Expand 10.8\%; aux tools & Physical protection \\
RISD (Design) & Maintain; creative pivot & Near tipping point \\
\bottomrule
\end{tabular}
\end{table}

\subsection{Policy Implications}

\textbf{For Institutions}: Differentiate responses by occupation; monitor tipping points; integrate AI literacy; emphasize human complementarities.

\textbf{For Policymakers}: Invest in physical-task occupations; support creative-sector transitions; fund verification capacity; enable continuous learning.

\textbf{For Students}: Assess physical protection; evaluate substitution ratios; develop meta-skills; plan for continuous evolution.

\subsection{Closing Remarks}

The Tipping Point Paradox---AI designed for augmentation triggering displacement---is resolved through the substitution ratio mechanism. Critically, tipping points are \textit{not fixed}: skill development, task recomposition, and verification capacity investment can extend the augmentation window. \textbf{The tipping point is not a cliff to be feared but a threshold to be understood, monitored, and---through thoughtful intervention---extended.}



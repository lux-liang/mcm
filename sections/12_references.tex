% ========== 12. References ==========
% Note: Section title and TOC entry are auto-generated by redefined thebibliography

\begin{thebibliography}{99}

\bibitem{eloundou2023gpts}
Eloundou, T., Manning, S., Mishkin, P., \& Rock, D. (2023).
GPTs are GPTs: An early look at the labor market impact potential of large language models.
\textit{arXiv preprint arXiv:2303.10130}.

\bibitem{bass1969diffusion}
Bass, F. M. (1969).
A new product growth for model consumer durables.
\textit{Management Science}, 15(5), 215--227.

\bibitem{argote2011organizational}
Argote, L., \& Miron-Spektor, E. (2011).
Organizational learning: From experience to knowledge.
\textit{Organization Science}, 22(5), 1123--1137.

\bibitem{frey2017automation}
Frey, C. B., \& Osborne, M. A. (2017).
The future of employment: How susceptible are jobs to computerisation?
\textit{Technological Forecasting and Social Change}, 114, 254--280.

\bibitem{autor2015paradox}
Autor, D. H. (2015).
Why are there still so many jobs? The history and future of workplace automation.
\textit{Journal of Economic Perspectives}, 29(3), 3--30.

\bibitem{mckinsey2023ai}
McKinsey \& Company. (2023).
The state of AI in 2023: Generative AI's breakout year.
\textit{McKinsey Global Survey}. Retrieved from \url{https://www.mckinsey.com/capabilities/quantumblack/our-insights/the-state-of-ai-in-2023-generative-ais-breakout-year}

\bibitem{bls2024oews}
Bureau of Labor Statistics. (2024).
Occupational Employment and Wage Statistics.
U.S. Department of Labor. Retrieved from \url{https://www.bls.gov/oes/}

\bibitem{onet2024}
O*NET Resource Center. (2024).
O*NET OnLine Database (Version 28.3).
U.S. Department of Labor. Retrieved from \url{https://www.onetonline.org/}

\bibitem{ipeds2024}
National Center for Education Statistics. (2024).
Integrated Postsecondary Education Data System (IPEDS).
U.S. Department of Education. Retrieved from \url{https://nces.ed.gov/ipeds/}

\end{thebibliography}


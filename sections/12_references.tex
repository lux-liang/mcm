% ========== 12. References ==========
\section*{References}
\addcontentsline{toc}{section}{References}

\begin{thebibliography}{99}

\bibitem{eloundou2023gpts}
Eloundou, T., Manning, S., Mishkin, P., \& Rock, D. (2023).
GPTs are GPTs: An early look at the labor market impact potential of large language models.
\textit{arXiv preprint arXiv:2303.10130}.

\bibitem{bass1969diffusion}
Bass, F. M. (1969).
A new product growth for model consumer durables.
\textit{Management Science}, 15(5), 215--227.

\bibitem{argote2011organizational}
Argote, L., \& Miron-Spektor, E. (2011).
Organizational learning: From experience to knowledge.
\textit{Organization Science}, 22(5), 1123--1137.

\bibitem{frey2017automation}
Frey, C. B., \& Osborne, M. A. (2017).
The future of employment: How susceptible are jobs to computerisation?
\textit{Technological Forecasting and Social Change}, 114, 254--280.

\bibitem{acemoglu2020ai}
Acemoglu, D., \& Restrepo, P. (2020).
Robots and jobs: Evidence from US labor markets.
\textit{Journal of Political Economy}, 128(6), 2188--2244.

\bibitem{autor2015paradox}
Autor, D. H. (2015).
Why are there still so many jobs? The history and future of workplace automation.
\textit{Journal of Economic Perspectives}, 29(3), 3--30.

\bibitem{brynjolfsson2014second}
Brynjolfsson, E., \& McAfee, A. (2014).
\textit{The Second Machine Age: Work, Progress, and Prosperity in a Time of Brilliant Technologies}.
W. W. Norton \& Company.

\bibitem{nedelkoska2018automation}
Nedelkoska, L., \& Quintini, G. (2018).
Automation, skills use and training.
\textit{OECD Social, Employment and Migration Working Papers}, No. 202.

\bibitem{manyika2017future}
Manyika, J., et al. (2017).
Jobs lost, jobs gained: Workforce transitions in a time of automation.
\textit{McKinsey Global Institute Report}.

\bibitem{arntz2016risk}
Arntz, M., Gregory, T., \& Zierahn, U. (2016).
The risk of automation for jobs in OECD countries.
\textit{OECD Social, Employment and Migration Working Papers}, No. 189.

\bibitem{webb2020impact}
Webb, M. (2020).
The impact of artificial intelligence on the labor market.
\textit{Stanford University Working Paper}.

\bibitem{felten2021occupational}
Felten, E., Raj, M., \& Seamans, R. (2021).
Occupational, industry, and geographic exposure to artificial intelligence.
\textit{Strategic Management Journal}, 42(12), 2195--2217.

\bibitem{bls2024oews}
Bureau of Labor Statistics. (2024).
Occupational Employment and Wage Statistics.
U.S. Department of Labor. Retrieved from \url{https://www.bls.gov/oes/}

\bibitem{onet2024}
O*NET Resource Center. (2024).
O*NET OnLine Database (Version 28.3).
U.S. Department of Labor. Retrieved from \url{https://www.onetonline.org/}

\bibitem{ipeds2024}
National Center for Education Statistics. (2024).
Integrated Postsecondary Education Data System (IPEDS).
U.S. Department of Education. Retrieved from \url{https://nces.ed.gov/ipeds/}

\bibitem{wef2023future}
World Economic Forum. (2023).
\textit{The Future of Jobs Report 2023}.
Geneva: World Economic Forum.

\bibitem{saltelli2008global}
Saltelli, A., et al. (2008).
\textit{Global Sensitivity Analysis: The Primer}.
John Wiley \& Sons.

\bibitem{jevons1865coal}
Jevons, W. S. (1865).
\textit{The Coal Question}.
Macmillan and Co.

\bibitem{sorrell2009jevons}
Sorrell, S. (2009).
Jevons' Paradox revisited: The evidence for backfire from improved energy efficiency.
\textit{Energy Policy}, 37(4), 1456--1469.

\end{thebibliography}


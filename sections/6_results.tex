% ========== 6. Results ==========
\Needspace{8\baselineskip}
\section{Simulation Results and Validation}

We present simulation results for the three target occupations over a 60-month horizon.

\subsection{Adoption and Employment Dynamics}

\begin{figure}[!htbp]
\centering
\includegraphics[width=0.85\textwidth]{fig07_calibration_fan.pdf}
\caption{Employment Trajectories with Uncertainty Quantification (60-month simulation). Shaded regions show 50\%, 80\%, and 95\% confidence intervals from Monte Carlo parameter sampling.}
\label{fig:employment_trajectories}
\end{figure}

\textbf{Employment Outcomes}: Information Security Analysts show strong growth (+15.8\% at $t=60$); Electricians exhibit modest gains (+2.6\%); Graphic Designers remain near baseline (+0.2\%) with high uncertainty.

\begin{figure}[!htbp]
\centering
\includegraphics[width=0.85\textwidth]{fig04_employment_surface.pdf}
\caption{Employment Response Surface: $\Delta E(\%)$ as function of AI adoption ceiling ($A_{cap}$) and substitution ratio. Occupation positions marked.}
\label{fig:employment_surface}
\end{figure}

\begin{table}[H]
\centering
\caption{Employment Outcomes at $t=60$ months}
\label{tab:employment_outcomes}
\small
\begin{tabular}{lcccc}
\toprule
\textbf{Occupation} & \textbf{$A(60)$} & \textbf{$E(60)/E(0)$} & \textbf{$\rho_{\max}$} & \textbf{$V_s(60)$} \\
\midrule
Information Security & 70.0\% & 1.158 (+15.8\%) & 0.29 & 0.688 \\
Electricians & 9.5\% & 1.026 (+2.6\%) & 0.03 & 0.475 \\
Graphic Designers & 85.0\% & 1.002 (+0.2\%) & 0.34 & 0.717 \\
\bottomrule
\end{tabular}
\end{table}

The divergence (+15.8\% vs.\ +0.2\%) despite similar high adoption rates underscores the role of substitution ratios, physical protection, and demand rebound in mediating AI impacts.

\subsection{Verification Congestion}

\begin{figure}[!htbp]
\centering
\includegraphics[width=0.85\textwidth]{fig03_congestion_heatmap.pdf}
\caption{Verification Congestion Ratio Evolution}
\label{fig:congestion}
\end{figure}

Info Security peaks at $\rho = 0.29$ (month 36), triggering adoption slowdown; Electricians remain below 0.03 (minimal constraint); Graphic Designers reach 0.34 by month 48, creating quality pressure but remaining below critical thresholds.

\subsection{Phase Portrait and Dynamical Regimes}

\begin{figure}[!htbp]
\centering
\includegraphics[width=0.80\textwidth]{fig06_phase_portrait.pdf}
\caption{Phase Portrait: Adoption--Employment Dynamics. Vector fields illustrate trajectory flows; nullclines partition the state space into distinct dynamical regimes.}
\label{fig:phase_portrait}
\end{figure}

The phase portrait (Fig.~\ref{fig:phase_portrait}) reveals three basins of attraction: (i) \textit{Augmentation} ($\dot{A} > 0$, $\dot{E} > 0$), (ii) \textit{Transition} ($\dot{A} > 0$, $\dot{E} \approx 0$), and (iii) \textit{Displacement} ($A \to A_{\text{cap}}$, $\dot{E} < 0$). Information Security converges to the Augmentation basin; Electricians remain confined to low-adoption regions; Graphic Designers traverse all three regimes as adoption approaches the 85\% ceiling. The basin structure, not instantaneous parameters, determines long-run equilibria.

\subsection{Calibration Framework and Epistemic Boundaries}

We employ a three-phase approach that explicitly separates what historical data can identify from what remains scenario-dependent.

\subsubsection{Phase 1: Structural Parameter Calibration (2010--2022)}

Pre-GenAI structural parameters ($\kappa_E$, $\delta_0$, $\varepsilon_D$) are calibrated by minimizing weighted SSE between model trajectory and BLS OEWS annual employment (2010--2022).

\begin{table}[H]
\centering
\caption{Structural Calibration Results (2010--2022)}
\label{tab:validation_phase1}
\small
\begin{tabular}{lccccc}
\toprule
\textbf{Occupation} & \textbf{BLS CAGR} & \textbf{Model CAGR} & \textbf{MAPE} & \textbf{$\kappa_E$} & \textbf{$\varepsilon_D$} \\
\midrule
Information Security & 7.0\% & 6.8\% & 5.2\% & 0.048 & 0.72 \\
Electricians & 2.5\% & 2.4\% & 4.8\% & 0.018 & 0.45 \\
Graphic Designers & 0.8\% & 0.9\% & 6.1\% & 0.012 & 0.58 \\
\bottomrule
\end{tabular}
\end{table}

\textit{MAPE below 7\% confirms successful capture of pre-GenAI dynamics; calibrated parameters define the counterfactual trajectory absent GenAI disruption.}

\subsubsection{Phase 2: GenAI Scenario Injection (2023+)}

GenAI parameters cannot be identified from current data (18-month window, confounded by tech layoffs and monetary tightening). We treat them as \textbf{scenario assumptions}:

\begin{table}[H]
\centering
\small
\begin{tabular}{llll}
\toprule
\textbf{Parameter} & \textbf{Scenario Value} & \textbf{Range Tested} & \textbf{Basis} \\
\midrule
$\kappa_A$ (adoption rate) & $1.5\times$ baseline & $[1.0, 2.5]\times$ & McKinsey 2023~\cite{mckinsey2023ai} \\
$A_{\text{cap}}$ (ceiling) & Occupation-specific & $\pm 20\%$ & Task exposure (Sec.~5.1) \\
$s_{\text{base}}$ (substitution) & Occupation-specific & $\pm 30\%$ & Eloundou et al.~\cite{eloundou2023gpts} \\
\bottomrule
\end{tabular}
\end{table}
\noindent\textit{Note: Ranges represent plausible bounds; no claim of empirical identification is made. All tested via sensitivity analysis (Sec.~9).}

\subsubsection{Phase 3: Directional Consistency Check (2023--2024)}

With only two post-GenAI data points, we conduct a \textit{sanity check}: do observed employment changes from the pre-GenAI baseline (2022) through 2024 fall within model prediction intervals?

\begin{table}[H]
\centering
\caption{Post-GenAI Consistency: Observed vs.\ Predicted (2022$\to$2024)}
\label{tab:consistency_check}
\small
\begin{tabular}{lcccc}
\toprule
\textbf{Occupation} & \textbf{BLS $\Delta$E$^\dagger$} & \textbf{Model 90\% CI} & \textbf{Direction} & \textbf{Consistent?} \\
\midrule
Information Security & +14.8\% & [+11\%, +22\%] & $\uparrow$ & Yes \\
Electricians & +12.2\% & [+4\%, +15\%] & $\uparrow$ & Yes \\
Graphic Designers & +5.4\% & [$-2$\%, +9\%] & $\uparrow$ & Yes \\
\bottomrule
\end{tabular}
\end{table}
\noindent\textit{$^\dagger$BLS $\Delta$E computed as $(E_{2024} - E_{2022})/E_{2022}$: ISA $(187940-163690)/163690$; Elec $(773950-690050)/690050$; GD $(223240-211890)/211890$. Source: Table~\ref{tab:audit_bls}.}

All occupations show directional consistency. Wide intervals reflect scenario uncertainty. Full epistemic classification of parameters is provided in Table~\ref{tab:audit_params} (Appendix A).

\subsection{Scenario Analysis}

\begin{figure}[!htbp]
\centering
\includegraphics[width=0.90\textwidth]{fig11_scenario_multiples.pdf}
\caption{Scenario Analysis: Employment Under Alternative Assumptions}
\label{fig:scenarios}
\end{figure}

\textbf{Scenario A} (Accelerated AI): Doubling $\kappa_A$ advances tipping points 12--18 months. \textbf{Scenario B} (Enhanced Verification): 50\% increase in $V_{h,\max}$ delays congestion. \textbf{Scenario C} (High Demand Rebound): $\varepsilon_D = 0.8$ shifts GD from +0.2\% to +2.5\%. \textbf{Scenario D} (Educational Intervention): $\phi = 0.30$ shifts GD's $A^*$ rightward by 12\%--15\%.

\FloatBarrier


% ========== 6. Results ==========
\Needspace{8\baselineskip}
\section{Simulation Results and Validation}

We present simulation results for the three target occupations over a 60-month horizon.

\subsection{Adoption and Employment Dynamics}

\begin{figure}[!htbp]
\centering
\includegraphics[width=0.85\textwidth]{fig07_calibration_fan.pdf}
\caption{Employment Trajectories with Uncertainty Quantification (60-month simulation). Shaded regions show 50\%, 80\%, and 95\% confidence intervals from Monte Carlo parameter sampling.}
\label{fig:employment_trajectories}
\end{figure}

\textbf{Employment Outcomes}: Information Security Analysts show strong growth (+15.8\% at $t=60$); Electricians exhibit modest gains (+2.6\%); Graphic Designers remain near baseline (+0.2\%) with high uncertainty.

\begin{figure}[!htbp]
\centering
\includegraphics[width=0.85\textwidth]{fig04_employment_surface.pdf}
\caption{Employment Response Surface: $\Delta E(\%)$ as function of AI adoption ceiling ($A_{cap}$) and substitution ratio. Occupation positions marked.}
\label{fig:employment_surface}
\end{figure}

\begin{table}[H]
\centering
\caption{Employment Outcomes at $t=60$ months}
\label{tab:employment_outcomes}
\small
\begin{tabular}{lcccc}
\toprule
\textbf{Occupation} & \textbf{$A(60)$} & \textbf{$E(60)/E(0)$} & \textbf{$\rho_{\max}$} & \textbf{$V_s(60)$} \\
\midrule
Information Security & 70.0\% & 1.158 (+15.8\%) & 0.29 & 0.688 \\
Electricians & 9.5\% & 1.026 (+2.6\%) & 0.03 & 0.475 \\
Graphic Designers & 85.0\% & 1.002 (+0.2\%) & 0.34 & 0.717 \\
\bottomrule
\end{tabular}
\end{table}

The divergence (+15.8\% vs.\ +0.2\%) despite similar high adoption rates underscores the role of substitution ratios, physical protection, and demand rebound in mediating AI impacts.

\subsection{Verification Congestion}

\begin{figure}[!htbp]
\centering
\includegraphics[width=0.85\textwidth]{fig03_congestion_heatmap.pdf}
\caption{Verification Congestion Ratio Evolution}
\label{fig:congestion}
\end{figure}

Info Security peaks at $\rho = 0.29$ (month 36), triggering adoption slowdown; Electricians remain below 0.03 (minimal constraint); Graphic Designers reach 0.34 by month 48, creating quality pressure but remaining below critical thresholds.

\subsection{Phase Portrait and Dynamical Regimes}

\begin{figure}[!htbp]
\centering
\includegraphics[width=0.80\textwidth]{fig06_phase_portrait.pdf}
\caption{Phase Portrait: Adoption--Employment Dynamics. Vector fields illustrate trajectory flows; nullclines partition the state space into distinct dynamical regimes.}
\label{fig:phase_portrait}
\end{figure}

The phase portrait (Fig.~\ref{fig:phase_portrait}) elucidates the topological structure of the $(A, E)$ state space. The vector field reveals three distinct basins of attraction: (i) the \textit{Augmentation basin}, characterized by trajectories where both $\dot{A} > 0$ and $\dot{E} > 0$, converging toward a stable fixed point with elevated employment; (ii) the \textit{Transition manifold}, a separatrix region where $\dot{A} > 0$ but $\dot{E} \approx 0$, representing the marginal zone near $A^*$; and (iii) the \textit{Displacement basin}, where trajectories approach the adoption ceiling ($A \to A_{\text{cap}}$) while employment contracts ($\dot{E} < 0$).

Information Security Analysts exhibit asymptotic stability within the Augmentation basin---their initial conditions and parameter configuration ensure convergence to a high-employment equilibrium regardless of perturbations. Electricians, constrained by physical protection ($P_i = 0.78$), remain confined to a low-adoption region where displacement dynamics never activate. In stark contrast, Graphic Designers' trajectory traverses all three regimes: initial augmentation gives way to transition near $A \approx 0.70$, with the system ultimately gravitating toward the Displacement basin as adoption approaches the 85\% ceiling. This topological analysis reveals why identical AI capabilities produce divergent labor market outcomes---the basin structure, not merely instantaneous parameter values, determines long-run equilibria.

\subsection{Calibration Framework and Epistemic Boundaries}

We employ a three-phase approach that explicitly separates what historical data can identify from what remains scenario-dependent.

\subsubsection{Phase 1: Structural Parameter Calibration (2010--2022)}

The pre-GenAI period permits identification of \textit{structural parameters} governing labor market dynamics. These parameters---employment elasticity ($\kappa_E$), natural attrition ($\delta_0$), demand responsiveness ($\varepsilon_D$)---characterize how labor markets respond to productivity shocks \textit{regardless of the specific technology}. The 12-year calibration window provides sufficient variation for robust estimation.

\textbf{Calibration targets}: Annual employment levels $E(t)$ from BLS OEWS; wage trends where available.

\textbf{Calibration method}: Minimize weighted sum of squared errors between model trajectory and observed data, with regularization to prevent overfitting.

\begin{table}[H]
\centering
\caption{Structural Calibration Results (2010--2022)}
\label{tab:validation_phase1}
\small
\begin{tabular}{lccccc}
\toprule
\textbf{Occupation} & \textbf{BLS CAGR} & \textbf{Model CAGR} & \textbf{MAPE} & \textbf{$\kappa_E$} & \textbf{$\varepsilon_D$} \\
\midrule
Information Security & 7.0\% & 6.8\% & 5.2\% & 0.048 & 0.72 \\
Electricians & 2.5\% & 2.4\% & 4.8\% & 0.018 & 0.45 \\
Graphic Designers & 0.8\% & 0.9\% & 6.1\% & 0.012 & 0.58 \\
\bottomrule
\end{tabular}
\end{table}

\textit{Interpretation}: MAPE below 7\% indicates the model successfully captures pre-GenAI employment dynamics. These calibrated parameters establish the counterfactual---what trajectories would obtain \textit{absent} GenAI disruption---against which scenario projections are compared.

\subsubsection{Phase 2: GenAI Scenario Injection (2023+)}

GenAI-specific parameters cannot be identified from current data. The 18-month post-ChatGPT window is insufficient given: (i) enterprise adoption lags consumer awareness by 12--24 months; (ii) BLS data reflects conditions 6--12 months prior to release; (iii) macroeconomic confounders (tech layoffs, monetary tightening) obscure technology-specific effects.

We therefore treat GenAI parameters as \textbf{scenario assumptions}:

\begin{table}[H]
\centering
\small
\begin{tabular}{llll}
\toprule
\textbf{Parameter} & \textbf{Scenario Value} & \textbf{Range Tested} & \textbf{Basis} \\
\midrule
$\kappa_A$ (adoption rate) & $1.5\times$ baseline & $[1.0, 2.5]\times$ & McKinsey AI Survey 2023 \\
$A_{\text{cap}}$ (ceiling) & Occupation-specific & $\pm 20\%$ & Task exposure analysis \\
$s_{\text{base}}$ (substitution) & Occupation-specific & $\pm 30\%$ & Expert elicitation \\
\bottomrule
\end{tabular}
\end{table}

\textit{Purpose}: These scenarios enable policy robustness analysis---identifying interventions that perform well across plausible futures---rather than point prediction.

\subsubsection{Phase 3: Directional Consistency Check (2023--2024)}

With only two post-GenAI data points, we conduct a \textit{sanity check}: do observed 2023--2024 employment changes fall within model prediction intervals? This tests whether model outputs are grossly inconsistent with emerging data, not whether GenAI parameters are correctly specified.

\begin{table}[H]
\centering
\caption{Post-GenAI Directional Consistency (2023--2024)}
\label{tab:consistency_check}
\small
\begin{tabular}{lcccc}
\toprule
\textbf{Occupation} & \textbf{BLS $\Delta$E} & \textbf{Model 90\% CI} & \textbf{Direction} & \textbf{Consistent?} \\
\midrule
Information Security & +14.8\% & [+11\%, +22\%] & $\uparrow$ & \checkmark \\
Electricians & +12.2\% & [+4\%, +15\%] & $\uparrow$ & \checkmark \\
Graphic Designers & +5.4\% & [$-2$\%, +9\%] & $\uparrow$ & \checkmark \\
\bottomrule
\end{tabular}
\end{table}

All occupations show directional consistency. Wide intervals reflect scenario uncertainty; convergence to tighter bounds requires additional post-GenAI years.

\subsubsection{Epistemic Status Summary}

\begin{table}[H]
\centering
\caption{Epistemic Status of Model Components}
\label{tab:epistemic_status}
\small
\begin{tabular}{p{3.5cm}lll}
\toprule
\textbf{Quantity} & \textbf{Status} & \textbf{Basis} & \textbf{Used In} \\
\midrule
$E(t)$ historical & Observed & BLS OEWS & Calibration target \\
$\kappa_E$, $\delta_0$, $\varepsilon_D$ & Calibrated & 2010--22 fit & All projections \\
$\Xi_o$, $P_i$ & Derived & O*NET decomposition & Exposure scores \\
$A_{\text{cap}}$, $\kappa_A$, $s(t)$ & Assumed (scenario) & Surveys, expert input & Scenario analysis \\
$A^*$ (tipping point) & Model output & $= 1/(2s)$ & Policy thresholds \\
$E(t)$ projected & Conditional forecast & If scenarios hold & Recommendations \\
\bottomrule
\end{tabular}
\end{table}

\noindent\textit{The model provides scenario-conditional projections, not unconditional predictions. Its value lies in identifying structural relationships---tipping points, congestion thresholds, physical protection effects---that govern AI-employment dynamics across a range of plausible futures.}

\subsection{Scenario Analysis}

\begin{figure}[!htbp]
\centering
\includegraphics[width=0.90\textwidth]{fig11_scenario_multiples.pdf}
\caption{Scenario Analysis: Employment Under Alternative Assumptions}
\label{fig:scenarios}
\end{figure}

\textbf{Scenario A} (Accelerated AI): Doubling $\kappa_A$ advances tipping points 12--18 months. \textbf{Scenario B} (Enhanced Verification): 50\% increase in $V_{h,\max}$ delays congestion. \textbf{Scenario C} (High Demand Rebound): $\varepsilon_D = 0.8$ shifts GD from +0.2\% to +2.5\%. \textbf{Scenario D} (Educational Intervention): $\phi = 0.30$ shifts GD's $A^*$ rightward by 12\%--15\%.

\FloatBarrier


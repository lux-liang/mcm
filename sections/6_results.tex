% ========== 6. Results ==========
\section{Simulation Results and Validation}

We present simulation results for the three target occupations over a 60-month horizon.

\subsection{Adoption and Employment Dynamics}

\begin{figure}[!htbp]
\centering
\includegraphics[width=0.85\textwidth]{fig07_calibration_fan.pdf}
\caption{Employment Trajectories with Uncertainty Quantification (60-month simulation). Shaded regions show 50\%, 80\%, and 95\% confidence intervals from Monte Carlo parameter sampling.}
\label{fig:employment_trajectories}
\end{figure}

\textbf{Employment Outcomes}: Information Security Analysts show strong growth (+17.5\% at $t=60$); Electricians exhibit modest gains (+2.9\%); Graphic Designers remain near baseline (+0.2\%) with high uncertainty.

\begin{figure}[!htbp]
\centering
\includegraphics[width=0.85\textwidth]{fig04_employment_surface.pdf}
\caption{Employment Response Surface: $\Delta E(\%)$ as function of AI adoption ceiling ($A_{cap}$) and substitution ratio. Occupation positions marked.}
\label{fig:employment_surface}
\end{figure}

\begin{table}[H]
\centering
\caption{Employment Outcomes at $t=60$ months}
\label{tab:employment_outcomes}
\small
\begin{tabular}{lcccc}
\toprule
\textbf{Occupation} & \textbf{$A(60)$} & \textbf{$E(60)/E(0)$} & \textbf{$\rho_{\max}$} & \textbf{$V_s(60)$} \\
\midrule
Information Security & 70.0\% & 1.175 (+17.5\%) & 0.32 & 0.697 \\
Electricians & 9.3\% & 1.029 (+2.9\%) & 0.15 & 0.489 \\
Graphic Designers & 85.0\% & 1.002 (+0.2\%) & 0.40 & 0.725 \\
\bottomrule
\end{tabular}
\end{table}

The divergence (+17.5\% vs.\ +0.2\%) despite similar adoption rates underscores the role of substitution ratios, physical protection, and demand rebound in mediating AI impacts.

\subsection{Verification Congestion}

\begin{figure}[!htbp]
\centering
\includegraphics[width=0.85\textwidth]{fig03_congestion_heatmap.pdf}
\caption{Verification Congestion Ratio Evolution}
\label{fig:congestion}
\end{figure}

Info Security peaks at $\rho = 0.32$ (month 36), triggering adoption slowdown; Electricians remain below 0.15 (no constraint); Graphic Designers exceed 0.40 by month 48, creating quality degradation.

\subsection{Phase Portrait and Tipping Points}

\begin{figure}[!htbp]
\centering
\includegraphics[width=0.80\textwidth]{fig06_phase_portrait.pdf}
\caption{Phase Portrait: Adoption--Employment Dynamics}
\label{fig:phase_portrait}
\end{figure}

The phase portrait reveals three regimes: \textit{Augmentation} (both $A$, $E$ increasing), \textit{Transition} ($A$ increasing, $E$ stabilizing), \textit{Displacement} ($A$ at ceiling, $E$ declining). Graphic Designers traverse all three; Electricians remain in augmentation.

\textbf{Tipping points} from $A^*(t) = 1/(2 \cdot \text{sub\_ratio}(t))$: at $t=0$, ISA 200\% (unreachable), EL 500\% (unreachable), GD 90.9\% (near ceiling). With skill growth, GD's effective $A^*$ increases to $\sim$119\% by $t=60$.

\subsection{Validation Strategy}

\subsubsection{Two-Phase Validation}

We separate validation into pre-GenAI trend calibration and post-shock stress testing:

\textbf{Phase 1: Pre-Trend Calibration (2010--2022).} Parameters are calibrated to reproduce BLS employment growth during the traditional automation era, prior to GenAI emergence.

\begin{table}[H]
\centering
\caption{Phase 1 Validation: Pre-GenAI Trend Fit (2010--2022)}
\label{tab:validation_phase1}
\small
\begin{tabular}{lccc}
\toprule
\textbf{Occupation} & \textbf{BLS CAGR} & \textbf{Model CAGR} & \textbf{MAPE} \\
\midrule
Information Security & 6.2\% & 5.9\% & 4.8\% \\
Electricians & 1.5\% & 1.6\% & 6.7\% \\
Graphic Designers & $-0.2\%$ & $-0.1\%$ & 8.5\% \\
\bottomrule
\end{tabular}
\end{table}

\textbf{Phase 2: Post-Shock Stress Test (2023--2024).} Model initialized at 2022 state, run forward with $\kappa_A$ scaled by $1.5\times$ (adoption acceleration post-ChatGPT). Predictions compared to preliminary 2023--2024 BLS estimates.

\textit{Result:} Employment trends remain within model 90\% CI for all three occupations; Graphic Designers show closest approach to CI boundary, consistent with model sensitivity predictions.

\subsubsection{Structural Robustness}

\textbf{Shock Timing.} Varying GenAI onset $t_{\text{shock}} \in$ [2022Q4, 2023Q2]: regime classifications unchanged; transition timing shifts 2--4 months.

\textbf{Breakpoint Test.} Chow-type structural break test at 2022Q4: $F$-statistics of 2.34 (ISA), 1.12 (EL), 3.87 (GD). For GD, marginally significant ($p < 0.10$), consistent with earlier AI-design tool adoption.

\subsubsection{Validation Limitations}

\begin{itemize}[noitemsep]
\item BLS data provide \textit{macro-consistency checks}, not causal identification; confounders not isolated.
\item GenAI window (2022--2024) is short; long-term validation requires future data.
\item Adoption rates $A(t)$ are modeled, not directly observed; proxy validation uses survey data with uncertainty.
\end{itemize}

The validation supports model plausibility for scenario analysis, but should not be interpreted as causal verification.

\subsection{Scenario Analysis}

\begin{figure}[!htbp]
\centering
\includegraphics[width=0.90\textwidth]{fig11_scenario_multiples.pdf}
\caption{Scenario Analysis: Employment Under Alternative Assumptions}
\label{fig:scenarios}
\end{figure}

\textbf{Scenario A} (Accelerated AI): Doubling $\kappa_A$ advances tipping points 12--18 months. \textbf{Scenario B} (Enhanced Verification): 50\% increase in $V_{h,\max}$ delays congestion. \textbf{Scenario C} (High Demand Rebound): $\varepsilon_D = 0.8$ shifts GD from +0.2\% to +2.8\%. \textbf{Scenario D} (Educational Intervention): $\phi = 0.30$ shifts GD's $A^*$ rightward by 12\%--15\%.

\FloatBarrier


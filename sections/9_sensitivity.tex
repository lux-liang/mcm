% ========== 9. Sensitivity Analysis ==========
\section{Sensitivity Analysis}

\subsection{Methodology}

We employ Partial Rank Correlation Coefficient (PRCC) analysis with Latin Hypercube Sampling (LHS) to assess global parameter sensitivity. PRCC is appropriate because: (1) it captures monotonic relationships under rank transformation, controlling for correlations among inputs; (2) it remains robust when output distributions are non-normal; (3) it quantifies each parameter's unique contribution after removing linear effects of others.

\textbf{Parameter Set and Perturbation.} We perturb $p = 8$ parameters around occupation-specific calibrated baselines (Table~\ref{tab:parameters}). Each parameter $\theta_j$ is sampled from $\theta_j^{\text{base}} \times (1 + \epsilon_j)$, where $\epsilon_j \sim \mathcal{U}(-r_j, r_j)$ with $r_j$ denoting the perturbation half-range. LHS ensures stratified coverage of the 8-dimensional input space with $N=1000$ samples, improving variance estimation over simple Monte Carlo.

\textbf{Output Metrics.} For each sample, we record: (i) terminal employment ratio $E(60)/E(0)$, (ii) maximum congestion $\rho_{\max}$, (iii) time to reach 50\% of adoption ceiling $t_{50}$. PRCC is computed separately for each occupation-output pair.

\begin{table}[H]
\centering
\caption{Parameter Perturbation Ranges for Sensitivity Analysis}
\label{tab:sens_params}
\small
\begin{tabular}{llc}
\toprule
\textbf{Parameter} & \textbf{Description} & \textbf{Perturbation} \\
\midrule
$A_{\text{cap}}$ & Adoption ceiling & $\pm 20\%$ \\
sub\_ratio$(t)$ & Substitution ratio & $\pm 30\%$ \\
$\kappa_A$ & Adoption rate & $\pm 40\%$ \\
$\kappa_E$ & Employment elasticity & $\pm 50\%$ \\
$\theta_V$ & Effective verification load & $\pm 30\%$ \\
$\gamma_G$ & Congestion superlinearity & $\pm 25\%$ \\
$\varepsilon_D$ & Demand elasticity & $\pm 40\%$ \\
$\beta_s$ & Skill-substitution elasticity & $\pm 30\%$ \\
\bottomrule
\end{tabular}
\end{table}

\textit{Note: Baseline values are occupation-specific; see Table~\ref{tab:parameters}. Perturbations are multiplicative around each occupation's calibrated baseline.}

\subsection{Global Sensitivity Results}

\begin{figure}[!htbp]
\centering
\includegraphics[width=0.80\textwidth]{fig08_sensitivity_heatmap.pdf}
\caption{Global Sensitivity Analysis (PRCC Coefficients)}
\label{fig:sensitivity_heatmap}
\end{figure}

\textbf{High-Sensitivity Parameters} ($|PRCC| > 0.7$): $A_{\text{cap}}$ (0.95 for adoption), sub\_ratio (-0.82 for employment), $\kappa_A$ (0.82), $\kappa_E$ (0.78), $\varepsilon_D$ (0.68 for employment via demand rebound).

\textbf{Low-Sensitivity Parameters}: $p_{\text{base}}$, $\delta_0$---limited impact due to AI-driven dynamics dominance.

\subsection{One-at-a-Time Sensitivity}

\begin{table}[H]
\centering
\caption{Employment Outcomes Under Parameter Variation}
\small
\begin{tabular}{lccc}
\toprule
\textbf{Scenario} & \textbf{Info Sec} & \textbf{Elec} & \textbf{Design} \\
\midrule
$A_{\text{cap}}$ $-20\%$ & +14.2\% & +2.5\% & +3.8\% \\
Baseline & +17.5\% & +2.9\% & +0.2\% \\
$A_{\text{cap}}$ $+20\%$ & +19.8\% & +3.2\% & $-4.5\%$ \\
\midrule
sub\_ratio $-30\%$ & +21.2\% & +3.4\% & +8.5\% \\
sub\_ratio $+30\%$ & +12.8\% & +2.4\% & $-9.2\%$ \\
\midrule
$\varepsilon_D = 0.3$ (low rebound) & +15.8\% & +2.6\% & $-1.2\%$ \\
$\varepsilon_D = 0.8$ (high rebound) & +19.1\% & +3.1\% & +2.8\% \\
\bottomrule
\end{tabular}
\end{table}

Graphic Designers show greatest sensitivity to both $A_{\text{cap}}$ and $\varepsilon_D$: a 20\% ceiling increase shifts employment from +0.2\% to $-4.5\%$; conversely, high demand elasticity ($\varepsilon_D = 0.8$) restores positive growth via the Jevons effect.

\subsection{Parameter Uncertainty}

Monte Carlo simulation (10,000 draws) yields 95\% confidence intervals:

\begin{table}[H]
\centering
\small
\begin{tabular}{lcc}
\toprule
\textbf{Occupation} & \textbf{Mean $E(60)/E(0)$} & \textbf{95\% CI} \\
\midrule
Info Security & 1.175 & [1.12, 1.23] \\
Electricians & 1.029 & [1.01, 1.05] \\
Graphic Designers & 1.002 & [0.92, 1.08] \\
\bottomrule
\end{tabular}
\end{table}

Graphic Designers exhibit the widest CI, reflecting uncertainty that could push outcomes from +8\% to $-8\%$.

\subsection{Critical Thresholds and Policy Implications}

The sensitivity analysis identifies parameter boundaries beyond which regime transitions occur:

\begin{itemize}[noitemsep]
\item \textbf{Graphic Designers displacement}: sub\_ratio $> 0.60$ OR $\varepsilon_D < 0.25$
\item \textbf{Electricians digital transformation}: $A_{\text{cap}} > 0.40$
\item \textbf{Info Security saturation}: $\kappa_A > 0.50$
\end{itemize}

The dominance of $A_{\text{cap}}$ (PRCC = 0.95) as the highest-sensitivity parameter carries profound policy implications. Effective intervention should prioritize mechanisms that modulate the \textit{adoption ceiling} rather than attempting to prohibit AI deployment outright. Regulatory approaches might include professional certification requirements for AI-assisted work, quality standards that necessitate human verification, or liability frameworks that incentivize human oversight. Simultaneously, investments in \textit{human verification capacity} ($V_{h,\max}$) offer a complementary lever: expanding cognitive bandwidth through training, workflow optimization, and human-AI interface improvements delays congestion onset and preserves the augmentation regime.

This sensitivity structure suggests that blunt instruments---technology bans, rigid automation taxes---will prove less effective than nuanced policies targeting the specific parameters ($A_{\text{cap}}$, $s(t)$, $V_{h,\max}$) that govern regime transitions. The goal should not be to halt technological diffusion but to shape its trajectory through the parameter space toward basins of attraction with favorable employment outcomes.

\FloatBarrier

